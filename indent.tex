\documentclass[12pt]{article}

% -------- # Packages

% Document layout and formatting
\usepackage[letterpaper, portrait, margin=1in, top=1.75in, bottom=1in, headheight=1in, heightrounded]{geometry}
\usepackage{fancyhdr, afterpage, etoolbox} 

% Fonts and text formatting
\usepackage[T1]{fontenc}
\usepackage{inconsolata, mathptmx}
\usepackage{xcolor, ragged2e, parskip}

% Math and calculations
\usepackage{amsmath, amsfonts, amssymb, amsthm, calc}

% Date formatting and utilities
\usepackage{datetime, lipsum}

% Graphics and lists
\usepackage{graphicx, enumitem}

% For table layout with wrapping
\usepackage{tabularx}

% -------- # Layout Adjustment

\newlength\newtop
\setlength{\newtop}{0.75in} % Set a custom top margin offset (0.75 inches)

\setlength{\parskip}{12pt} % Double spacing between paragraphs
\setlength{\parindent}{0pt} % No indentation for new paragraphs

% -------- # Custom Settings & Commands

\makeatletter
\patchcmd\@outputpage{\global \@colht \textheight}{%
    \global\textheight=\dimexpr\textheight+\newtop\relax%
    \global\topmargin=\dimexpr\topmargin-\newtop\relax%
    \global\@colht\textheight%
    \global\newtop\z@}{}{\err} % Reset \newtop after adjustment
\makeatother

% Define the extra space (two non-breaking spaces) as a length
\newcommand{\spaces}{~~}
\newlength{\extraSpaceLength}
\settowidth{\extraSpaceLength}{\spaces} % Calculate the width of the extra space

% Define a macro to calculate the column width dynamically with added space
\newcommand{\setcolumnwidth}[2]{
    \newlength{#1}%
    \settowidth{#1}{#2}%
    \addtolength{#1}{\extraSpaceLength} % Add extra space to the column width
}

% Define custom date format (DTG format in military writing)
\newdateformat{DTG}{%
  \twodigit{\THEDAY}~\monthname[\THEMONTH]~\THEYEAR} % Format: DD Month YYYY

% Define a custom color for the letterhead (Air Force branding)
\definecolor{headercolor}{HTML}{121292}

% Create a command for styled letterhead with the custom color
\newcommand{\letterhead}[1]{\textcolor{headercolor}{\textsc{#1}}}

% -------- # Header & Footer

\setlength{\footskip}{0.5in}

\fancyfoot[C]{
    \fontsize{8}{10}\selectfont
    \texttt{CONTROLLED UNCLASSIFIED INFORMATION (CUI) which must be protected under the Freedom of Information Act (5 U.S.C. 552) and/or the Privacy Act of 1974 (5 U.S.C. 552a). Unauthorized disclosure or misuse of this PERSONAL INFORMATION may result in disciplinary action, criminal and/or civil penalties.}
}

\renewcommand{\headrulewidth}{0pt}
\renewcommand{\footrulewidth}{0pt}

\fancypagestyle{default}{\fancyhead{}}

\fancypagestyle{firstpage}{
    \fancyhead[L]{\hspace*{-0.5in}\vspace*{-0.25in}\includegraphics[width=1in]{"DAF Seal.png"}} % Air Force seal aligned to left
    \fancyhead[C]{
        \letterhead{\textbf{DEPARTMENT OF THE AIR FORCE}}\\
        \letterhead{123d EXAMPLE MEMORANDUM SQUADRON (ACC)}\\
        \letterhead{FORT GEORGE G. MEADE, MARYLAND}
    }
}

% -------- # List settings

% First-level list (aligning with margin and wrapping properly)
\setlist[enumerate,1]{label=\arabic*., left=0pt, labelsep=\extraSpaceLength, itemindent=0pt, listparindent=0pt, itemsep=1em}

% Second-level list (indented first line, wraps to left margin)
\setlist[enumerate,2]{label=\alph*., left=2em, labelsep=\extraSpaceLength, itemindent=\extraSpaceLength, listparindent=0pt, itemsep=1em}

% Third-level list (further indented, same behavior)
\setlist[enumerate,3]{label=(\alph*), left=4em, labelsep=\extraSpaceLength, itemindent=\extraSpaceLength, listparindent=0pt, itemsep=1em}

% -------- # Page details

\begin{document}
\pagestyle{default}
\thispagestyle{firstpage}
\RaggedRight

% -------- # Begin Document

% DTG aligned to the right (AF memorandum standards)
\noindent\hfill\DTG\today

% MEMORANDUM FOR section
\setcolumnwidth{\memoWidth}{MEMORANDUM FOR}
\noindent
\begin{tabularx}{\textwidth}{@{} p{\memoWidth} @{} X @{}} 
MEMORANDUM FOR & TSGT JANE S. AUSTEN (1234567890)\\
               & SRA CHARLOTTE H. BRONTË (7890123456)\\
\end{tabularx}

% FROM: section
\setcolumnwidth{\fromWidth}{FROM:}
\noindent
\begin{tabularx}{\textwidth}{@{} p{\fromWidth} @{} X @{}} 
FROM: & TSgt Evelyn Aurora Lee\\
      & 123d Example Memorandum Squadron\\
      & 4567 Writing Street\\
      & Fort George G. Meade, MD 20755\\
\end{tabularx}

% SUBJECT: section
\setcolumnwidth{\subjectWidth}{SUBJECT:}
\noindent
\begin{tabularx}{\textwidth}{@{} p{\subjectWidth} @{} X @{}} 
SUBJECT: & Use of the Official Memorandum Template\\
\end{tabularx}

% REFERENCES section
\setcolumnwidth{\referencesWidth}{References:}
\noindent
\begin{tabularx}{\textwidth}{@{} p{\referencesWidth} @{} X @{}} 
References: & (a)~AFM 33-326, 25 November 2011, \textit{Preparing Official Communications.}\\
            & (b)~DoDM 5110.04-M-V2, October 26, 2010, \textit{DoD Manual for Written Material.}\\
            & (c)~DAFI 36-1245, September 12, 2024, \textit{Pretending You're a Famous Dead Author and Your Troop is Another Dead Author.}\\
\end{tabularx}

% Main content placeholder
\begin{enumerate}

    \item \lipsum[1] % First level (wraps to margin)
    
    \begin{enumerate}
        \item \lipsum[1] % Second level (indented first line, wraps to margin)

        \item \lipsum[1]
        
        \begin{enumerate}
            \item \lipsum[1] % Third level (indented first line, wraps to margin)
        \end{enumerate}

    \end{enumerate}
    
    \item \lipsum[1] % First level (wraps to margin)
    
\end{enumerate}

% -------- # End Document
\end{document}