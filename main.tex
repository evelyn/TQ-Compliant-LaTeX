\documentclass[12pt]{article}

% -------- # Packages
% Document layout and formatting
\usepackage[letterpaper, portrait, margin=1in, top=1.75in, bottom=1in, headheight=1in, heightrounded]{geometry}
\usepackage{fancyhdr, afterpage, etoolbox} 

% Fonts and text formatting
\usepackage[T1]{fontenc}
\usepackage{inconsolata, mathptmx}
\usepackage{xcolor, ragged2e, parskip}

% Math and calculations
\usepackage{amsmath, amsfonts, amssymb, amsthm, calc, pgffor}

% Date formatting and utilities
\usepackage{datetime}

% Graphics and lists
\usepackage{graphicx, enumitem}

% For table layout with wrapping and raggedright alignment
\usepackage{tabularx}

% -------- # Layout Adjustment
\newlength\newtop
\setlength{\newtop}{0.75in} % Set a custom top margin offset (0.75 inches)
\setlength{\parskip}{12pt} % Double spacing between paragraphs
\setlength{\parindent}{0pt} % No indentation for new paragraphs

\makeatletter
\patchcmd\@outputpage{\global \@colht \textheight}{%
    \global\textheight=\dimexpr\textheight+\newtop\relax%
    \global\topmargin=\dimexpr\topmargin-\newtop\relax%
    \global\@colht\textheight%
    \global\newtop\z@}{}{\err} % Reset \newtop after adjustment
\makeatother

% -------- # Custom Settings & Commands
% Define a single non-breaking space (~) as a base unit length
\newlength{\baseSpaceLength}
\settowidth{\baseSpaceLength}{~} % Set the base space length to the width of a single non-breaking space
\newcommand{\spaceLength}[1]{\dimexpr #1\baseSpaceLength\relax} % Define a command that multiplies the length of ~ by a factor
\newcommand{\indentLevel}[2]{\dimexpr #2\baseSpaceLength * #1\relax} % Define a command that multiplies the length of ~ by two factors

% Define a macro to calculate the column width dynamically with added space
\newcommand{\setcolumnwidth}[2]{
    \newlength{#1}%
    \settowidth{#1}{#2}%
    \addtolength{#1}{\spaceLength{2}} % Add extra space to the column width
}

% Define DTG date format
\newdateformat{DTG}{%
  \twodigit{\THEDAY}~\monthname[\THEMONTH]~\THEYEAR} % Format: DD Month YYYY

% Define a custom color for the letterhead (DAF branding)
\definecolor{headercolor}{HTML}{121292}

% Create a command for styled letterhead with the custom color
\newcommand{\letterhead}[1]{\textcolor{headercolor}{\textsc{#1}}}

% Create a command to apply list formatting settings for a specific level
\newcommand{\formatListLevel}[4]{
  \setlist[enumerate,#1]{
    label=#2, % The label format for the current list level
    itemindent=\dimexpr \indentLevel{#1}{#3}\relax, % Controls the left of the margin and the left of the item
    left=\spaceLength{0}, % Horizontal movement for everything
    labelsep=\dimexpr \spaceLength{#4}\relax, % \spaceLength{N} = ~ in label
    leftmargin=\spaceLength{0} % Align everything flush with the left margin
  }
}

% -------- # Header & Footer
\setlength{\footskip}{0.5in}

\fancyfoot[C]{
    \fontsize{8}{10}\selectfont
    \texttt{CONTROLLED UNCLASSIFIED INFORMATION (CUI) which must be protected under the Freedom of Information Act (5 U.S.C. 552) and/or the Privacy Act of 1974 (5 U.S.C. 552a). Unauthorized disclosure or misuse of this PERSONAL INFORMATION may result in disciplinary action, criminal and/or civil penalties.}
}

\renewcommand{\headrulewidth}{0pt}
\renewcommand{\footrulewidth}{0pt}

\fancypagestyle{default}{\fancyhead{}}

\fancypagestyle{firstpage}{
    \fancyhead[L]{\hspace*{-0.5in}\vspace*{-0.25in}\includegraphics[width=1in]{"DAF Seal.png"}} % Air Force seal aligned to left
    \fancyhead[C]{
        \letterhead{\textbf{DEPARTMENT OF THE AIR FORCE}}\\
        \letterhead{123d EXAMPLE MEMORANDUM SQUADRON (ACC)}\\
        \letterhead{FORT GEORGE G. MEADE, MARYLAND}
    }
}

% -------- # List settings
\formatListLevel{1}{\arabic*.}{4}{2}
\formatListLevel{2}{\alph*.}{4}{2}
\formatListLevel{3}{(\alph*)}{4}{2}
\formatListLevel{4}{\roman*.}{4}{2}
\formatListLevel{5}{(\Alph*)}{4}{2}
\formatListLevel{6}{(\Roman*)}{4}{2}

% -------- # Page details
\begin{document}
\pagestyle{default}
\thispagestyle{firstpage}
\begin{RaggedRight}

% DTG aligned to the right (AF memorandum standards)
\noindent\hfill\DTG\today

% MEMORANDUM FOR section
\setcolumnwidth{\memoWidth}{MEMORANDUM FOR}
\noindent
\begin{tabularx}{\textwidth}{@{} p{\memoWidth} @{} >{\RaggedRight}X @{}}
MEMORANDUM FOR & SSGT ADELINE V. WOOLF (9012345678)\\
               & SRA CHARLOTTE H. BRONTË (7890123456)\\
\end{tabularx}

% FROM: section
\setcolumnwidth{\fromWidth}{FROM:}
\noindent
\begin{tabularx}{\textwidth}{@{} p{\fromWidth} @{} >{\RaggedRight}X @{}}
FROM: & TSGT JANE S. AUSTEN (1234567890)\\
      & 123d Example Memorandum Squadron\\
      & 4567 Writing Street\\
      & Fort George G. Meade, MD 20755\\
\end{tabularx}

% SUBJECT: section
\setcolumnwidth{\subjectWidth}{SUBJECT:}
\noindent
\begin{tabularx}{\textwidth}{@{} p{\subjectWidth} @{} >{\RaggedRight}X @{}}
SUBJECT: & In Which We Attempt, Against All Odds and with Only a Slightly Better Than Average Chance of Success, to Explain the Intricate Mechanics of Formatting List Levels in LaTeX While Simultaneously Avoiding a Catastrophic Collapse of Space-Time and Maintaining a Stiff Upper Lip in the Face of Imminent Badness Levels, All for the Sake of Preserving the Number 42 (Which, We All Know, Is the Answer to Life, the Universe, and Everything)\\
\end{tabularx}

% REFERENCES section
\setcolumnwidth{\referencesWidth}{References:}
\noindent
\begin{tabularx}{\textwidth}{@{} p{\referencesWidth} @{} >{\RaggedRight}X @{}}
References: & (a)~AFM 70-420, 20 April 2011, \textit{Preparing Official Communications}.\\
            & (b)~DoDM 1235.77-M-V8, 09 November, 2010, \textit{DoD Manual for Writing Manuals on Writing Manuals}.\\
            & (c)~DAFI 42-1245, September 18, 2024, \textit{Pretending You're a Famous Dead Author and Your Troop is Another Dead Author}.\\
\end{tabularx}

% Official Memorandum Content
\begin{enumerate}
    \item Use only approved organizational letterhead for all correspondence. This applies to all letterhead, both pre-printed and computer generated. Reference (a) details the format and style of official letterhead such as centering the first line of the header 5/8ths of an inch from the top of the page in 12 point Copperplate Gothic Bold font (Letterhead Line 1 style of this template). The second header line is centered 3 points below the first line in 10.5 point Copperplate Gothic Bold font (Letterhead Line 2 style of this template).
    \item Note that this template provides proper formatting for various elements via Word Styles. The recipient line uses the “MEMO FOR” style, the body of the memorandum uses the “Body” style, the signature block uses the “Signature Block” style, and so on. The “Normal” style is left available for edge cases not covered by this template, but should be used sparingly, if at all.
    \item Place “MEMORANDUM FOR” on the second line below the date. Leave two spaces between “MEMORANDUM FOR” and the recipient’s office symbol. If there are multiple recipients, two or three office symbols may be placed on each line aligned under the entries on the first line. If there are numerous recipients, type “DISTRIBUTION” after “MEMORANDUM FOR” and use a “DISTRIBUTION” element below the signature block.
    \item Place “FROM:” on the second line below the “MEMORANDUM FOR” line. Leave two spaces between the colon in “FROM:” and the originator’s office symbol. The “FROM:” element contains the full mailing address of the originator’s office unless the mailing address is in the header or if all the recipients are located in the same installation as the originator.
    \item Place “SUBJECT:” in uppercase letters on the second line below the last line of the FROM element. Leave two spaces between the colon in “SUBJECT:” and the subject. Capitalize the first letter of each word except articles, prepositions, and conjunctions (this is sometimes referred to as “title case”). Be clear and concise. If the subject is long, try to revise and shorten the subject; if shortening is not feasible, align the second line under the first word of the subject.
    \item Body text begins on the second line below the last line in the subject element and is flush with the left margin. If the Reference element is used, then the body text begins on the second line below the last line on the Reference element.
        \begin{enumerate}
            \item When a paragraph is split between pages, there must be at least two lines from the paragraph on both pages. This template should automatically handle this formatting, but in the case of widowed sentences you can use a manual page break. Similarly, avoid single-sentence paragraphs by revising or reorganizing the content.
            \item Number or letter each body text paragraph and subparagraph according to the format for subdividing paragraphs in official memorandums presented in the Tongue and Quill. This subdivision format is provided in this template in the “Body” style. When a memorandum is subdivided, the number of levels used should be relative to the length of the memorandum. Shorter products typically use three or fewer levels. Longer products may use more levels, but only the number of levels needed.
        \end{enumerate}
    \item Follow the spacing guidance for between the text, signature block, attachment element, courtesy copy element, and distribution lists, if used, carefully. The signature block starts on the fifth line below the body text – this spacing is handled automatically by the Signature Block style. Never separate the text from the signature block: the signature page must include body text above the signature block. Also, the first element below the signature block begins on the third line below the last line of the duty title; this applies to attachments, courtesy copies, and distribution lists, whichever is used first.
    \item Elements of the Official Memorandum can be inserted as templated parts in MS Word. Select the “Insert” menu, “Quick Parts”, and then select the desired element.
        \begin{enumerate}
            \item To add classification banner markings (including FOUO markings), click on the header or footer, and in the “Header \& Footer Tools” tab, select the Header/Footer dropdown menus on the left side. Use the pre-generated headers and footers – note that the first page is different from the second page onward, and so the second page header and footer must be applied separately.
        \end{enumerate}

\newpage
    \item The example of this memorandum applies to many official memorandums that Airmen may be tasked to prepare; however, there are additional elements for special uses of the official memorandum. Refer to the Tongue and Quill discussion on the official memorandum for more details, or consult published guidance applicable to your duties.
\end{enumerate}

% Signature Block
\vspace{1in} % Adjust as needed
\noindent FIRST M. LAST, Rank, USAF\\
Duty Title
\newpage

% Attachments
\noindent X Attachment(s): (if none, delete element)\\
1. Attachment description, date\\
2. Attachments section should begin on the third line below the signature block.

% cc section
\noindent cc: Rank and name, ORG/SYMBOL, or both\\
The CC element should begin on the second line below the attachment element. If no attachment element is present, it begins on the third line below the signature block.

% DISTRIBUTION section
\noindent \textbf{DISTRIBUTION:} ORG/SYMBOL or Organization Name\\
The distribution element should begin on the second line below the attachment or CC element. If neither attachment nor CC elements are present, this element begins on the third line below the signature block.

% 1st Indorsement
\noindent 1st Ind, ORG/SYMBOL [Office symbol for 1st Indorsement official]\\

% MEMORANDUM FOR 2nd Indorsement
\noindent \textbf{MEMORANDUM FOR:} ORG/SYMBOL [Office symbol for 2d Indorsement official]

% Indorsement content
\begin{enumerate}
    \item Number each indorsement in sequence (1st Ind, 2d Ind, 3d Ind, etc.). Begin the first indorsement on the second line below the last element of the official memorandum. Begin subsequent indorsements on the second line below the last element of the previous indorsement. Follow the indorsement number with your office symbol.
    \item When inserting a new indorsement element (either on the same page or as a separate page), the numbering will need to be reset. This is accomplished by right-clicking the number on the list, then selecting “Restart at 1.”
    \item Use a separate page indorsement when there isn’t space remaining on the original memorandum or previous indorsement page. The separate-page indorsement is basically the same as the one for the same page, except the top line always cites the indorsement number with the originator’s office, date, and subject of the original communication; the second line reflects the functional address symbol of the indorsing office with the date.
\end{enumerate}

% Signature Block for 1st Indorsement
\vspace{1in} % Adjust as needed
\noindent FIRST M LAST, Rank, USAF\\
Duty Title\\
(Align w/ third character of 2nd line)

% End of page - Start new page for the next indorsement
\newpage
% 2nd Indorsement Page
\newpage
\noindent 2d Ind to [Originator ORG/SYMBOL], DD Mon YY, Memorandum Subject\\

\noindent ORG/SYMBOL [Office Symbol for 2d Indorsement official] \hfill DD Mon YYYY

% MEMORANDUM FOR section for 2nd Indorsement
\noindent \textbf{MEMORANDUM FOR:} ORG/SYMBOL [Originator]

% 2nd Indorsement content
\begin{enumerate}
    \item Use a separate page indorsement when there isn’t space remaining on the original memorandum or previous indorsement page. The separate-page indorsement is basically the same as the one for the same page, except the top line always cites the indorsement number with the originator’s office, date, and subject of the original communication; the second line reflects the functional address symbol of the indorsing office with the date.
\end{enumerate}

% Signature Block for 2nd Indorsement
\vspace{1in} % Adjust spacing as needed
\noindent FIRST M LAST, Rank, USAF\\
Duty Title\\
(Align w/ third character of 2nd line)

\end{RaggedRight}
\end{document}
